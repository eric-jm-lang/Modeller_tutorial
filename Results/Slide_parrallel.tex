\begin{frame}[fragile]
\frametitle{Loop modelling 3 - what is new?}
\begin{itemize}
\item We are going to use Model 9 as a starting point to optimise the second loop 
\newline
\item As this loop is significantly longer, calculations will also be longer
\newline
\item To reduce the computational time we will specify in the script that we want to run the modelling in parallel (i.e. using more than one processor core)
\newline
\item This is achieved by importing the \texttt{modeller.parallel} library and adding a few instructions within the script.
\newline
\item To be able to run the job in parallel, it is also required to save the \texttt{class MyLoop} statement into a separate python script that we are going to call \texttt{MyLoop.py}
\end{itemize}

\end{frame}

\begin{frame}[fragile]
\frametitle{Loop modelling 3 - script \texttt{MyLoop.py}}
\begin{lstlisting}
from modeller import *
from modeller.automodel import * 

class MyLoop(dopehr_loopmodel):
    def select_atoms(self):
        return selection(self.residue_range('218:', '231:'))
    def select_loop_atoms(self):
        return selection(self.residue_range('218:', '231:'))
\end{lstlisting}
\end{frame}

\begin{frame}[fragile]
\frametitle{Loop modelling 3 - script \texttt{loop\_modelling\_3.py}}
\begin{lstlisting}
from modeller import *
from modeller.automodel import * 
from modeller.parallel import * # This enables to work with multiple processor cores
from MyLoop import MyLoop # The MyLoop class is in a different file and thus needs to be imported 

j = job()
# The following specify the number of processor cores to use, do not specify more core than you have available on your machine
j.append(local_slave())
j.append(local_slave())
j.append(local_slave())
j.append(local_slave())
j.append(local_slave())
j.append(local_slave())

log.verbose()
env = environ()
env.io.atom_files_directory = ['.', '../atom_files']
\end{lstlisting}
\end{frame}

\begin{frame}[fragile]
\frametitle{Loop modelling 3 - script \texttt{loop\_modelling\_3.py} cont'd}
\begin{lstlisting}
# no need to refer to an alignment anymore are there, we just want to optimize the loop
a = MyLoop(env, inimodel='model9.pdb', # Best model form previous step
           sequence='model9', # code (same name but without '.pdb')
           loop_assess_methods=assess.DOPE) # assess with DOPE  
              
# Then the specified loop is refined and 25 models created
a.loop.starting_model = 1
a.loop.ending_model   = 25

# Very thorough optimization:
a.loop.library_schedule = autosched.slow
a.loop.max_var_iterations = 300
# Thorough MD optimization:
a.loop.md_level = refine.slow
# Repeat the whole optimization cycle 2 times
a.loop.repeat_optimization = 2
# Specify that the run should be in parrallel
a.use_parallel_job(j) 
a.make()
\end{lstlisting}
\end{frame}
